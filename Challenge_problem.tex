\documentclass[journal,12pt,twocolumn]{IEEEtran}
%
\usepackage{setspace}
\usepackage{gensymb}
%\doublespacing
\singlespacing

%\usepackage{graphicx}
%\usepackage{amssymb}
%\usepackage{relsize}
\usepackage[cmex10]{amsmath}
%\usepackage{amsthm}
%\interdisplaylinepenalty=2500
%\savesymbol{iint}
%\usepackage{txfonts}
%\restoresymbol{TXF}{iint}
%\usepackage{wasysym}
\usepackage{amsthm}
%\usepackage{iithtlc}
\usepackage{mathrsfs}
\usepackage{txfonts}
\usepackage{stfloats}
\usepackage{bm}
\usepackage{cite}
\usepackage{cases}
\usepackage{subfig}
%\usepackage{xtab}
\usepackage{longtable}
\usepackage{multirow}
%\usepackage{algorithm}
%\usepackage{algpseudocode}
\usepackage{enumitem}
\usepackage{mathtools}
\usepackage{steinmetz}
\usepackage{tikz}
\usepackage{circuitikz}
\usepackage{verbatim}
\usepackage{tfrupee}
\usepackage[breaklinks=true]{hyperref}
%\usepackage{stmaryrd}
\usepackage{tkz-euclide} % loads  TikZ and tkz-base
%\usetkzobj{all}
\usetikzlibrary{calc,math}
\usepackage{listings}
    \usepackage{color}                                            %%
    \usepackage{array}                                            %%
    \usepackage{longtable}                                        %%
    \usepackage{calc}                                             %%
    \usepackage{multirow}                                         %%
    \usepackage{hhline}                                           %%
    \usepackage{ifthen}                                           %%
  %optionally (for landscape tables embedded in another document): %%
    \usepackage{lscape}     
\usepackage{multicol}
\usepackage{chngcntr}
%\usepackage{enumerate}

%\usepackage{wasysym}
%\newcounter{MYtempeqncnt}
\DeclareMathOperator*{\Res}{Res}
%\renewcommand{\baselinestretch}{2}
\renewcommand\thesection{\arabic{section}}
\renewcommand\thesubsection{\thesection.\arabic{subsection}}
\renewcommand\thesubsubsection{\thesubsection.\arabic{subsubsection}}

\renewcommand\thesectiondis{\arabic{section}}
\renewcommand\thesubsectiondis{\thesectiondis.\arabic{subsection}}
\renewcommand\thesubsubsectiondis{\thesubsectiondis.\arabic{subsubsection}}

% correct bad hyphenation here
\hyphenation{op-tical net-works semi-conduc-tor}
\def\inputGnumericTable{}                                 %%

\lstset{
%language=C,
frame=single, 
breaklines=true,
columns=fullflexible
}
%\lstset{
%language=tex,
%frame=single, 
%breaklines=true
%}

\begin{document}
%


\newtheorem{theorem}{Theorem}[section]
\newtheorem{problem}{Problem}
\newtheorem{proposition}{Proposition}[section]
\newtheorem{lemma}{Lemma}[section]
\newtheorem{corollary}[theorem]{Corollary}
\newtheorem{example}{Example}[section]
\newtheorem{definition}[problem]{Definition}
%\newtheorem{thm}{Theorem}[section] 
%\newtheorem{defn}[thm]{Definition}
%\newtheorem{algorithm}{Algorithm}[section]
%\newtheorem{cor}{Corollary}
\newcommand{\BEQA}{\begin{eqnarray}}
\newcommand{\EEQA}{\end{eqnarray}}
\newcommand{\define}{\stackrel{\triangle}{=}}

\bibliographystyle{IEEEtran}
%\bibliographystyle{ieeetr}


\providecommand{\mbf}{\mathbf}
\providecommand{\pr}[1]{\ensuremath{\Pr\left(#1\right)}}
\providecommand{\qfunc}[1]{\ensuremath{Q\left(#1\right)}}
\providecommand{\sbrak}[1]{\ensuremath{{}\left[#1\right]}}
\providecommand{\lsbrak}[1]{\ensuremath{{}\left[#1\right.}}
\providecommand{\rsbrak}[1]{\ensuremath{{}\left.#1\right]}}
\providecommand{\brak}[1]{\ensuremath{\left(#1\right)}}
\providecommand{\lbrak}[1]{\ensuremath{\left(#1\right.}}
\providecommand{\rbrak}[1]{\ensuremath{\left.#1\right)}}
\providecommand{\cbrak}[1]{\ensuremath{\left\{#1\right\}}}
\providecommand{\lcbrak}[1]{\ensuremath{\left\{#1\right.}}
\providecommand{\rcbrak}[1]{\ensuremath{\left.#1\right\}}}
\theoremstyle{remark}
\newtheorem{rem}{Remark}
\newcommand{\sgn}{\mathop{\mathrm{sgn}}}
\providecommand{\abs}[1]{\left\vert#1\right\vert}
\providecommand{\res}[1]{\Res\displaylimits_{#1}} 
\providecommand{\norm}[1]{\left\lVert#1\right\rVert}
%\providecommand{\norm}[1]{\lVert#1\rVert}
\providecommand{\mtx}[1]{\mathbf{#1}}
\providecommand{\mean}[1]{E\left[ #1 \right]}
\providecommand{\fourier}{\overset{\mathcal{F}}{ \rightleftharpoons}}
%\providecommand{\hilbert}{\overset{\mathcal{H}}{ \rightleftharpoons}}
\providecommand{\system}{\overset{\mathcal{H}}{ \longleftrightarrow}}
	%\newcommand{\solution}[2]{\textbf{Solution:}{#1}}
\newcommand{\solution}{\noindent \textbf{Solution: }}
\newcommand{\cosec}{\,\text{cosec}\,}
\providecommand{\dec}[2]{\ensuremath{\overset{#1}{\underset{#2}{\gtrless}}}}
\newcommand{\myvec}[1]{\ensuremath{\begin{pmatrix}#1\end{pmatrix}}}
\newcommand{\mydet}[1]{\ensuremath{\begin{vmatrix}#1\end{vmatrix}}}
%\numberwithin{equation}{section}
\numberwithin{equation}{subsection}
%\numberwithin{problem}{section}
%\numberwithin{definition}{section}
\makeatletter
\@addtoreset{figure}{problem}
\makeatother

\let\StandardTheFigure\thefigure
\let\vec\mathbf
%\renewcommand{\thefigure}{\theproblem.\arabic{figure}}
\renewcommand{\thefigure}{\theproblem}
%\setlist[enumerate,1]{before=\renewcommand\theequation{\theenumi.\arabic{equation}}
%\counterwithin{equation}{enumi}


%\renewcommand{\theequation}{\arabic{subsection}.\arabic{equation}}

\def\putbox#1#2#3{\makebox[0in][l]{\makebox[#1][l]{}\raisebox{\baselineskip}[0in][0in]{\raisebox{#2}[0in][0in]{#3}}}}
     \def\rightbox#1{\makebox[0in][r]{#1}}
     \def\centbox#1{\makebox[0in]{#1}}
     \def\topbox#1{\raisebox{-\baselineskip}[0in][0in]{#1}}
     \def\midbox#1{\raisebox{-0.5\baselineskip}[0in][0in]{#1}}

\vspace{3cm}


\title{Challenge Problem 3}
\author{Jayati Dutta}





% make the title area
\maketitle

\newpage

%\tableofcontents

\bigskip

\renewcommand{\thefigure}{\theenumi}
\renewcommand{\thetable}{\theenumi}
%\renewcommand{\theequation}{\theenumi}


%\begin{abstract}
%This is a simple document explaining how to determine the QR decomposition of a 2x2 matrix.
%\end{abstract}

%Download all python codes 
%
%\begin{lstlisting}
%svn co https://github.com/JayatiD93/trunk/My_solution_design/codes
%\end{lstlisting}

%Download all and latex-tikz codes from 
%%
%\begin{lstlisting}
%svn co https://github.com/gadepall/school/trunk/ncert/geometry/figs
%\end{lstlisting}
%%


\section{Problem}
Prove that - Normal matrices are unitarily diagonalizable.

\section{Explanation}
Let $A$ be a normal matrix, then we have to prove A is unitary diagonalizable.\\
$\textbf{Definition:}$\\
$A$ is normal if $A A^* = A^* A$\\
$\textbf{Definition:}$\\
A is unitary diagonalizable if there is a unitary matrix $U$ and diagonal matrix $D$ such that $UAU^* = D$.\\
$\textbf{Proof:}$\\
As $A$ is normal, so $A A^* = A^* A$. Now, by mathematical induction first we will consider the orthonormal vectors for n=2.\\
Consider an eigen vector $U$ of A corresponds to the eigen value $\lambda$ and $U$ is unit vector. Now $V$ is considered in such a way that ${U, V}$ forms an orthonormal basis in $C^2$.\\
As $U$ is an eigen vector , so $AU=\lambda U$.
Now,
\begin{align}
AU = U \myvec{\lambda & U^{-1}AV\\0}\\
AU = U \myvec{\lambda & \alpha\\0 & \beta}\\
A = U \myvec{\lambda & \alpha\\0 & \beta} U^*
\end{align}
where $U^{-1}AV = \myvec{\alpha\\\beta}$
Since $A$ is normal, 
\begin{align}
A^* A = U \myvec{\bar{\lambda} & 0\\\bar{\alpha} & \bar{\beta}} U^* U \myvec{\lambda & \alpha\\0 & \beta} U^*\\
\implies A^* A = U \myvec{\bar{\lambda}\lambda & \alpha\bar{\lambda}\\\lambda\bar{\alpha} & \alpha\bar{\alpha}+\bar{\beta}\beta}U^*
\end{align}
Where $\bar{\lambda} = \lambda ^*$ is the conjugate of $\lambda$.
Similarly,
\begin{align}
A A^* =  U \myvec{\bar{\lambda}\lambda & \alpha\bar{\beta}\\\beta\bar{\alpha} & \bar{\beta}\beta}U^*
\end{align}
As $A A^* = A^* A$
\begin{align}
\alpha\bar{\alpha} = 0\\
\implies \norm{\alpha} = 0\\
\implies \alpha =0\\
\implies A = U \myvec{\lambda & 0\\0 & \beta} U^*\\
\implies A = U D U^*
\end{align}
So, A is unitary diagonalizable when $D$ is a 2x2 matrix.
Now, assume that the result holds for (n-1). we can claim that there is $y$ such that 
\begin{align}
AU = U\myvec{\lambda & y\\0 & B}
\end{align}
where $B \in C^{(n-1)\times(n-1)}$, we decompose the matrix $A$ into blocks and compute the products of $A A^*$ and $A^* A$ as follows:
\begin{align}
A A^* =  U \myvec{\lambda & y\\0 & B}\myvec{\bar{\lambda} & 0\\y^* & \bar{B}}U^*\\
\implies A A^* =  U \myvec{\bar{\lambda}\lambda + yy^* & y\bar{B}\\By^* & B\bar{B}} U^*\\
\implies A A^* = U \myvec{\norm{\lambda}^2 + \norm{y}^2 & y \bar{B}\\By^* & B\bar{B}} U^*
\end{align}
Where $\bar{y^*}$ is the conjugate transpose of $y^*$.
Now,
\begin{align}
A^* A = U \myvec{\bar{\lambda} & 0\\y^* & \bar{B}}\myvec{\lambda & y\\0 & B}U^*\\
\implies A^* A =  U \myvec{\bar{\lambda}\lambda & y \bar{\lambda}\\\lambda y^* & \bar{B}B + y^*y} U^*\\
\implies A^* A =U \myvec{\norm{\lambda}^2 & y \bar{\lambda}\\\lambda y^* & \bar{B}B + \norm{y}^2} U^*
\end{align}
As $A A^* = A^* A$, 
\begin{align}
y y^* = 0\\
\implies \norm{y}^2 = 0\\
\implies y =0\\
\implies A = U \myvec{\lambda & 0\\0 & B} U^*\\
\implies A = U D U^*
\end{align}
As $\bar{B} = B^*$ and $B^* B = B B^*$, so $B$ is also normal and $B \in C^{(n-1)\times(n-1)}$. So it must be diagonal by mathematical induction hypothesis.
Let, 
\begin{align}
B = M D_1 M^*
\end{align}
where $M$ is unitary matrix and $D_1$ is diagonal matrix and both are in $C^{(n-1)\times(n-1)}$.
Now,
\begin{align}
AU = U \myvec{\lambda & 0\\0 & M D_1 M^*}\\
AU = U \myvec{1 & 0\\0 & M}\myvec{\lambda & 0\\0 & D_1}\myvec{1 & 0\\0 & M^*}\\
AU\myvec{1 & 0\\0 & M} = U \myvec{1 & 0\\0 & M}\myvec{\lambda & 0\\0 & D_1}\\
AW = W\myvec{\lambda & 0\\0 & D_1}\\
A = W\myvec{\lambda & 0\\0 & D_1}W^*
\end{align}
where $W$ is also a unitary matrix and $W=U\myvec{1 & 0\\0 & M}$
This implies that if $A$ is normal then $A$ is unitary diagonalizable.
%\renewcommand{\theequation}{\theenumi}
%\begin{enumerate}[label=\thesection.\arabic*.,ref=\thesection.\theenumi]
%\numberwithin{equation}{enumi}
%\item Verification of the above problem using python code.\\
%\solution The  following Python code generates Fig. \ref{fig:parabola}
%\begin{lstlisting}
%codes/check_parab.py
%\end{lstlisting}
%
%So the solution is matching with the given plot, hence it is verified.
%%
%\end{enumerate}

\end{document}



